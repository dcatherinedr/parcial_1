% Options for packages loaded elsewhere
\PassOptionsToPackage{unicode}{hyperref}
\PassOptionsToPackage{hyphens}{url}
%
\documentclass[
]{article}
\usepackage{amsmath,amssymb}
\usepackage{lmodern}
\usepackage{iftex}
\ifPDFTeX
  \usepackage[T1]{fontenc}
  \usepackage[utf8]{inputenc}
  \usepackage{textcomp} % provide euro and other symbols
\else % if luatex or xetex
  \usepackage{unicode-math}
  \defaultfontfeatures{Scale=MatchLowercase}
  \defaultfontfeatures[\rmfamily]{Ligatures=TeX,Scale=1}
\fi
% Use upquote if available, for straight quotes in verbatim environments
\IfFileExists{upquote.sty}{\usepackage{upquote}}{}
\IfFileExists{microtype.sty}{% use microtype if available
  \usepackage[]{microtype}
  \UseMicrotypeSet[protrusion]{basicmath} % disable protrusion for tt fonts
}{}
\makeatletter
\@ifundefined{KOMAClassName}{% if non-KOMA class
  \IfFileExists{parskip.sty}{%
    \usepackage{parskip}
  }{% else
    \setlength{\parindent}{0pt}
    \setlength{\parskip}{6pt plus 2pt minus 1pt}}
}{% if KOMA class
  \KOMAoptions{parskip=half}}
\makeatother
\usepackage{xcolor}
\IfFileExists{xurl.sty}{\usepackage{xurl}}{} % add URL line breaks if available
\IfFileExists{bookmark.sty}{\usepackage{bookmark}}{\usepackage{hyperref}}
\hypersetup{
  pdftitle={Parcial\#1},
  pdfauthor={Diana Catherine Davila},
  hidelinks,
  pdfcreator={LaTeX via pandoc}}
\urlstyle{same} % disable monospaced font for URLs
\usepackage[margin=1in]{geometry}
\usepackage{color}
\usepackage{fancyvrb}
\newcommand{\VerbBar}{|}
\newcommand{\VERB}{\Verb[commandchars=\\\{\}]}
\DefineVerbatimEnvironment{Highlighting}{Verbatim}{commandchars=\\\{\}}
% Add ',fontsize=\small' for more characters per line
\usepackage{framed}
\definecolor{shadecolor}{RGB}{248,248,248}
\newenvironment{Shaded}{\begin{snugshade}}{\end{snugshade}}
\newcommand{\AlertTok}[1]{\textcolor[rgb]{0.94,0.16,0.16}{#1}}
\newcommand{\AnnotationTok}[1]{\textcolor[rgb]{0.56,0.35,0.01}{\textbf{\textit{#1}}}}
\newcommand{\AttributeTok}[1]{\textcolor[rgb]{0.77,0.63,0.00}{#1}}
\newcommand{\BaseNTok}[1]{\textcolor[rgb]{0.00,0.00,0.81}{#1}}
\newcommand{\BuiltInTok}[1]{#1}
\newcommand{\CharTok}[1]{\textcolor[rgb]{0.31,0.60,0.02}{#1}}
\newcommand{\CommentTok}[1]{\textcolor[rgb]{0.56,0.35,0.01}{\textit{#1}}}
\newcommand{\CommentVarTok}[1]{\textcolor[rgb]{0.56,0.35,0.01}{\textbf{\textit{#1}}}}
\newcommand{\ConstantTok}[1]{\textcolor[rgb]{0.00,0.00,0.00}{#1}}
\newcommand{\ControlFlowTok}[1]{\textcolor[rgb]{0.13,0.29,0.53}{\textbf{#1}}}
\newcommand{\DataTypeTok}[1]{\textcolor[rgb]{0.13,0.29,0.53}{#1}}
\newcommand{\DecValTok}[1]{\textcolor[rgb]{0.00,0.00,0.81}{#1}}
\newcommand{\DocumentationTok}[1]{\textcolor[rgb]{0.56,0.35,0.01}{\textbf{\textit{#1}}}}
\newcommand{\ErrorTok}[1]{\textcolor[rgb]{0.64,0.00,0.00}{\textbf{#1}}}
\newcommand{\ExtensionTok}[1]{#1}
\newcommand{\FloatTok}[1]{\textcolor[rgb]{0.00,0.00,0.81}{#1}}
\newcommand{\FunctionTok}[1]{\textcolor[rgb]{0.00,0.00,0.00}{#1}}
\newcommand{\ImportTok}[1]{#1}
\newcommand{\InformationTok}[1]{\textcolor[rgb]{0.56,0.35,0.01}{\textbf{\textit{#1}}}}
\newcommand{\KeywordTok}[1]{\textcolor[rgb]{0.13,0.29,0.53}{\textbf{#1}}}
\newcommand{\NormalTok}[1]{#1}
\newcommand{\OperatorTok}[1]{\textcolor[rgb]{0.81,0.36,0.00}{\textbf{#1}}}
\newcommand{\OtherTok}[1]{\textcolor[rgb]{0.56,0.35,0.01}{#1}}
\newcommand{\PreprocessorTok}[1]{\textcolor[rgb]{0.56,0.35,0.01}{\textit{#1}}}
\newcommand{\RegionMarkerTok}[1]{#1}
\newcommand{\SpecialCharTok}[1]{\textcolor[rgb]{0.00,0.00,0.00}{#1}}
\newcommand{\SpecialStringTok}[1]{\textcolor[rgb]{0.31,0.60,0.02}{#1}}
\newcommand{\StringTok}[1]{\textcolor[rgb]{0.31,0.60,0.02}{#1}}
\newcommand{\VariableTok}[1]{\textcolor[rgb]{0.00,0.00,0.00}{#1}}
\newcommand{\VerbatimStringTok}[1]{\textcolor[rgb]{0.31,0.60,0.02}{#1}}
\newcommand{\WarningTok}[1]{\textcolor[rgb]{0.56,0.35,0.01}{\textbf{\textit{#1}}}}
\usepackage{graphicx}
\makeatletter
\def\maxwidth{\ifdim\Gin@nat@width>\linewidth\linewidth\else\Gin@nat@width\fi}
\def\maxheight{\ifdim\Gin@nat@height>\textheight\textheight\else\Gin@nat@height\fi}
\makeatother
% Scale images if necessary, so that they will not overflow the page
% margins by default, and it is still possible to overwrite the defaults
% using explicit options in \includegraphics[width, height, ...]{}
\setkeys{Gin}{width=\maxwidth,height=\maxheight,keepaspectratio}
% Set default figure placement to htbp
\makeatletter
\def\fps@figure{htbp}
\makeatother
\setlength{\emergencystretch}{3em} % prevent overfull lines
\providecommand{\tightlist}{%
  \setlength{\itemsep}{0pt}\setlength{\parskip}{0pt}}
\setcounter{secnumdepth}{-\maxdimen} % remove section numbering
\ifLuaTeX
  \usepackage{selnolig}  % disable illegal ligatures
\fi

\title{Parcial\#1}
\author{Diana Catherine Davila}
\date{2022-03-24}

\begin{document}
\maketitle

\begin{enumerate}
\def\labelenumi{\arabic{enumi}.}
\tightlist
\item
  La tabla mostrada contiene valores de poblacion, en cientos de miles,
  de las diez ciudades mas pobladas de cuatros paises en el año 1967:
\end{enumerate}

\begin{enumerate}
\def\labelenumi{\alph{enumi}.}
\tightlist
\item
  Construya un box-plot e identifique los puntos extremos ¿ Cuales son
  las caracteristicas? ¿Hay outliers?
\end{enumerate}

\begin{Shaded}
\begin{Highlighting}[]
\FunctionTok{par}\NormalTok{(}\AttributeTok{mfrow=} \FunctionTok{c}\NormalTok{(}\DecValTok{1}\NormalTok{,}\DecValTok{1}\NormalTok{))}
\CommentTok{\# Grafica bloxpot de argentina }
\FunctionTok{boxplot}\NormalTok{(arg,  }\AttributeTok{col=}\StringTok{"blue"}\NormalTok{, }\AttributeTok{main =} \StringTok{"Poblacion Argentina"}\NormalTok{, }\AttributeTok{xlab =} \StringTok{"Poblacion"}\NormalTok{, }\AttributeTok{ylim =} \FunctionTok{c}\NormalTok{(}\DecValTok{0}\NormalTok{,}\DecValTok{30}\NormalTok{),  }\AttributeTok{horizontal =} \ConstantTok{TRUE}\NormalTok{)}
\end{Highlighting}
\end{Shaded}

\includegraphics{parcial1_files/figure-latex/unnamed-chunk-1-1.pdf}

\begin{Shaded}
\begin{Highlighting}[]
\CommentTok{\#Grafica bloxport de USA}
\FunctionTok{boxplot}\NormalTok{(eeuu,  }\AttributeTok{col=}\StringTok{"red"}\NormalTok{, }\AttributeTok{main =} \StringTok{"Estados Unidos"}\NormalTok{, }\AttributeTok{xlab =} \StringTok{"Poblacion"}\NormalTok{, }\AttributeTok{ylim =} \FunctionTok{c}\NormalTok{(}\DecValTok{0}\NormalTok{,}\DecValTok{80}\NormalTok{),  }\AttributeTok{horizontal =} \ConstantTok{TRUE}\NormalTok{)}
\end{Highlighting}
\end{Shaded}

\includegraphics{parcial1_files/figure-latex/unnamed-chunk-1-2.pdf}

\begin{Shaded}
\begin{Highlighting}[]
\CommentTok{\#Grafica bloxpot de Holanda}
\FunctionTok{boxplot}\NormalTok{(hol,  }\AttributeTok{col=}\StringTok{"yellow"}\NormalTok{, }\AttributeTok{main =} \StringTok{"Poblacion Holanda"}\NormalTok{, }\AttributeTok{xlab =} \StringTok{"Poblacion"}\NormalTok{, }\AttributeTok{ylim =} \FunctionTok{c}\NormalTok{(}\DecValTok{0}\NormalTok{,}\DecValTok{10}\NormalTok{),  }\AttributeTok{horizontal =} \ConstantTok{TRUE}\NormalTok{)}
\end{Highlighting}
\end{Shaded}

\includegraphics{parcial1_files/figure-latex/unnamed-chunk-1-3.pdf}

\begin{Shaded}
\begin{Highlighting}[]
\CommentTok{\#Grafica bloxpot de Japón}
\FunctionTok{boxplot}\NormalTok{(jap,  }\AttributeTok{col=}\StringTok{"purple"}\NormalTok{, }\AttributeTok{main =} \StringTok{"Poblacion Japon"}\NormalTok{, }\AttributeTok{xlab =} \StringTok{"Poblacion"}\NormalTok{, }\AttributeTok{ylim =} \FunctionTok{c}\NormalTok{(}\DecValTok{0}\NormalTok{,}\DecValTok{115}\NormalTok{),  }\AttributeTok{horizontal =} \ConstantTok{TRUE}\NormalTok{)}
\end{Highlighting}
\end{Shaded}

\includegraphics{parcial1_files/figure-latex/unnamed-chunk-1-4.pdf} R/:
Se puede observar que la grafica que corresponde a Japón, Estados Unidos
y Argentina son los unicos que tiene datos atipicos, los cuales son para
de 110.21(Japon), 77.81(Estados Unidos), 29.66(Argentina),
adicionalmente la grafica de Argentina, Estados Unidos y Holanda tienen
una simetria positiva y la grafica de Japon es la unica que tiene una
simetria normal.

\begin{enumerate}
\def\labelenumi{\alph{enumi}.}
\setcounter{enumi}{1}
\tightlist
\item
  Compare los centros de cada población, sus dispersiones y su simetria.
  ¿ Cuales son las caracteristicas más sobresalientes? ¿ Hay outliers?
\end{enumerate}

R/: la grafica de japon es la unica que tiene una simetria normal, los
datos son centrados y solo tiene un outlier de 110,21

la grafica de Argentina tiene una simetria positiva, su centro esta a
25\%, tiene un solo dato atipico que es de 29.66

La grafica de Holanda tiene una simetria positiva, es la que tiene la
caja más grande este nos indica que tiene datos muy alejados su centro
esta aproximadamente a un 30\% y no posee datos atipicos.

La grafica de Estados Unidos tiene una simetria positiva, tiene su
centro a un 35\%, tiene un dato atipico de 77.81

\begin{enumerate}
\def\labelenumi{\arabic{enumi}.}
\setcounter{enumi}{1}
\tightlist
\item
  Avianca se encuentra estudiando la situacion de venta de puesto en el
  vuelo Armmenia - Bogota puesto que ha identificado que el 2\% de las
  personas que reservaron puesto, no se presentan a la hora del
  embarque. Teniendo en cuenta que el avion A320 utilizado en esta ruta
  tiene una capacidad para 189 pasajeros, la empresa decide vender 181
  tiquetes. ¿Cual es la probabilidad de que todas las personas que
  llegan a embarque puedan tener su silla asegurada y no se presenten
  quejar por ``sobre-vender'' sillas ante la aeronautica civil? Asuma
  que las llegadas de los pasajerons siguen una distribucion Poisson.
\end{enumerate}

\begin{Shaded}
\begin{Highlighting}[]
\NormalTok{lambda }\OtherTok{\textless{}{-}}\DecValTok{180} \SpecialCharTok{*} \FloatTok{0.98}
\FunctionTok{ppois}\NormalTok{(}\DecValTok{180}\NormalTok{, }\AttributeTok{lambda =}\NormalTok{ lambda, }\AttributeTok{lower.tail =} \ConstantTok{TRUE}\NormalTok{)}
\end{Highlighting}
\end{Shaded}

\begin{verbatim}
## [1] 0.6255245
\end{verbatim}

R/: La probabilidad de que todas las personas que llegan a embarque y
puedan tener su silla asegurada y no presenten quejas por sobre-vender
sillas ante la aeronautica civil es de aproximadamente de 62\%

\begin{enumerate}
\def\labelenumi{\arabic{enumi}.}
\setcounter{enumi}{2}
\tightlist
\item
  En la planta envasadora de Coca-Cola FEMSA ubicada en Tocancipá se
  realiza el embotellamiento de Coca-Cola presentación de 400ML (en
  promedio), con una desviacion estandar de 5ML. Si se sabe que por
  experiencia que este proceso de embotellado sigue una distribucion
  normal, y tambien se save que todo producto con más de 415 ML es
  declarado NO CONFORME. Determine el porcentaje de las botellas de
  gaseosa que son declaradas producto NO CONFORME
\end{enumerate}

\begin{Shaded}
\begin{Highlighting}[]
\NormalTok{med }\OtherTok{\textless{}{-}} \DecValTok{400}

\NormalTok{dtsn }\OtherTok{\textless{}{-}} \DecValTok{5}

\NormalTok{resp }\OtherTok{\textless{}{-}} \FunctionTok{pnorm}\NormalTok{(}\DecValTok{415}\NormalTok{,med,dtsn,}\AttributeTok{lower.tail =} \ConstantTok{FALSE}\NormalTok{)}

\FunctionTok{print}\NormalTok{(}\FunctionTok{paste}\NormalTok{(}\StringTok{"El porcentaje de botellas producto NO CONFORME es de: "}\NormalTok{,resp}\SpecialCharTok{*}\DecValTok{100}\NormalTok{,}\StringTok{\textquotesingle{}\%\textquotesingle{}}\NormalTok{));}
\end{Highlighting}
\end{Shaded}

\begin{verbatim}
## [1] "El porcentaje de botellas producto NO CONFORME es de:  0.134989803163009 %"
\end{verbatim}

\end{document}
